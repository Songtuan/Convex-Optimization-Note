\documentclass[10pt,a4paper]{article}
\usepackage[utf8]{inputenc}
\usepackage{amsmath}
\usepackage{amsfonts}
\usepackage{amssymb}
\usepackage[left=2cm,right=2cm,top=2cm,bottom=2cm]{geometry}
\author{Garrick Lin}
\title{Convex Optimization Note}
\setlength{\parindent}{0cm}
\setlength{\parskip}{15pt}
\begin{document}
\maketitle

\section{Affine Set}

\subsection{Definition}
A set $\mathcal{A}$ is called affine set when it satisfied:
\begin{quotation}
	If $\mathbf{x_{1}} \in \mathcal{A}$ and $\mathbf{x_{2}} \in \mathcal{A}$, then, $\forall \theta \in \mathcal{R}$, $x = \theta \mathbf{x_{1}} + (1 - \theta) \mathbf{x_{2}}$ also belong to $\mathcal{A}$. 
\end{quotation}
The expression $\theta \mathbf{x_{1}} + (1 - \theta) \mathbf{x_{2}}$ represent a line that cross through $\mathbf{x_{1}}$ and $\mathbf{x_{2}}$. Hence, affine set can be explained intuitively as:
\begin{quote}
	Affine set is a set that contain the line which cross through any two point within this set.
\end{quote} 

\subsection{Properties}
Assume a affine set $\mathcal{A} \subseteq \mathcal{R}^{n}$. If $\mathcal{A}$ contain original point($\mathbf{0}$), then, $\mathcal{A}$ is a subspace. In order to prove this, $\mathcal{A}$ must satisfied following three rules:  
\begin{enumerate}
	\item $\mathcal{A}$ contain original point.
	\item If $\mathbf{v} \in \mathcal{A}$, then, $\forall \theta \in \mathcal{R}$, $\theta \mathbf{v} \in \mathcal{A}$.
	\item If $\mathbf{v_{1}}, \mathbf{v_{2}} \in \mathcal{A}$, then, $\mathbf{v_{1}} + \mathbf{v_{2}} \in \mathbf{A}$.
\end{enumerate}
The first rule is already satisfied. For the second rule, $\forall \theta \in \mathcal{R}$, $\theta \mathbf{v}$ satisfied:
\begin{equation*}
	\theta \mathbf{v} = \theta \mathbf{v} + (1 - \theta) \mathbf{0}
\end{equation*}
Since $\mathcal{A}$ is affine and $\mathbf{v}, \mathbf{0} \in \mathcal{A}$, $\theta \mathbf{v} \in \mathcal{A}$ always true. For the third rule, there is:
\begin{equation*}
	\mathbf{v_{1}} + \mathbf{v_{2}} = 2(\frac{1}{2} \mathbf{v_{1}} + \frac{1}{2} \mathbf{v_{2}})
\end{equation*}
$\frac{1}{2} \mathbf{v_{1}} + \frac{1}{2} \mathbf{v_{2}}$ is in $\mathcal{A}$ as $\mathcal{A}$ is affine and $\frac{1}{2} \mathbf{v_{1}}, \frac{1}{2} \mathbf{v_{2}} \in \mathcal{A}$(According to the second rule, $\mathbf{v_{1}}, \mathbf{v_{2}} \in \mathcal{A}$, then, $\frac{1}{2} \mathbf{v_{1}}, \frac{1}{2} \mathbf{v_{2}} \in \mathcal{A}$). According  to the second rule, $2(\frac{1}{2} \mathbf{v_{1}} + \frac{1}{2} \mathbf{v_{2}}) \in \mathcal{A}$

Therefore, any affine set $\mathcal{A}'$ can be thought as a subspace $\mathcal{V}$ with a transportation $\mathbf{v}$, which means, $\mathcal{A}' = \mathcal{V} + \mathbf{v}$. On the other hand, an affine set subtract some constant vector ($\mathcal{A}' - \mathbf{v}$), which make the new set contain original point, can form a subspace. Formally, for a affine set $\mathcal{A} = \{ \mathbf{a_{1}}, \mathbf{a_{2}}, \cdots, \mathbf{a_{n}} \}$ and  $\mathbf{a_{i}} \in \mathcal{A}$, the set:
\begin{equation}
	\mathcal{A} - \mathbf{a_{i}} = \{ \mathbf{a_{1}} - \mathbf{a_{i}}, \mathbf{a_{2}} - \mathbf{a_{i}}, \cdots,  \mathbf{a_{i - 1}}, \mathbf{0}, \mathbf{a_{i + 1}}, \cdots, \mathbf{a_{n}} - \mathbf{a_{i}} \}
	\label{affine_subspace}
\end{equation}
form a subspace. Furthermore, any affine set $\mathcal{A}$ has the form:
\begin{equation}
	\mathcal{A} = \{ \mathbf{x} |\, \mathcal{B} \mathbf{x} = \mathbf{c} \}
	\label{affine_matrix}
\end{equation}
\textbf{Proof}: According to equation \ref{affine_subspace}, $\mathcal{A} - \mathbf{a_{i}}$ is a subspace. Denote this subspace as $\mathcal{L} = \mathcal{A} - \mathbf{a_{i}}$. Assume the space that perpendicular to $\mathcal{L}$, denote as $\mathcal{L}^{\perp}$, has basis $\{ \mathbf{b_{1}}, \mathbf{b_{2}}, \cdots, \mathbf{b_{n}} \}$. Then, $\mathcal{L}$ is the set of vectors that perpendicular to the basis of $\mathcal{L}^{\perp}$. As a result, if we write the basis of $\mathcal{L}^{\perp}$ in matrix form, which is:
\begin{equation*}
	\mathcal{B} = 
	\begin{bmatrix}
		\mathbf{b_{1}}^{T} \\
		\mathbf{b_{2}}^{T} \\
		\vdots \\
		\mathbf{b_{n}}^{T}
	\end{bmatrix}
\end{equation*}
We have:
\begin{equation}
	\mathcal{L} = \{ \mathbf{y} |\, \mathcal{B} \mathbf{y} = 0 \}
	\label{affine_original_subspace}
\end{equation} 
Equation \ref{affine_original_subspace} always hold as all vector in $\mathcal{L}^{\perp}$ is the linear combination of the basis, hence, if a vector $\mathbf{v}$ satisfied $\mathbf{b_{1}}^{T} \mathbf{v} = 0, \mathbf{b_{2}}^{T} \mathbf{v} = 0, \cdots, \mathbf{b_{n}}^{T} \mathbf{v} = 0$, then, $(\theta_{1} \mathbf{b_{1}} + \theta_{2} \mathbf{b_{2}} + \cdots \theta_{n} \mathbf{b_{n}})^{T} \mathbf{v} = 0$, which means,$\mathbf{v}$ is perpendicular to any vector in $\mathcal{L}^{\perp}$ if $\mathbf{v}$ is perpendicular to the basis of $\mathcal{L}^{\perp}$.

Since the vector in $\mathcal{L}$ come from $\mathcal{A}$ subtract $\mathbf{a_{i}}$, for all $\mathbf{y} \in \mathcal{L}$, there must exists $\mathbf{x} \in \mathcal{A}$ such that: 
\begin{equation}
	\mathbf{y} = \mathbf{x} - \mathbf{a_{i}}
	\label{affine_subtract}
\end{equation}
By combining equation \ref{affine_subtract} and equation \ref{affine_original_subspace}, we have:
\begin{equation*}
	\mathcal{A} = \{ \mathbf{x} |\, \mathcal{B} (\mathbf{x} - \mathbf{a_{i}}) = 0 \}
\end{equation*}
Which is equivalent to :
\begin{equation*}
	\mathcal{A} = \{ \mathbf{x} |\, \mathcal{B} \mathbf{x} = \mathcal{B} \mathbf{a_{i}} \}
\end{equation*}
and has the same form as equation \ref{affine_matrix}, where $\mathbf{c} = \mathcal{B} \mathbf{a_{i}}$.

\section{Hyper-plane}
In a $\mathcal{R}^{n}$ space, the hyperplane is defined as a $n - 1$ dimension subspace plus a translate $\mathbf{x_{0}}$, which is perpendicular to a one dimension vector $\mathbf{w}$. The formal mathematical expression of hyper-plane $\mathcal{H}$ is:
\begin{equation}
	\mathcal{H} = \{ \mathbf{y} + \mathbf{x_{0}} |\, \mathbf{w}^{T}\mathbf{y} = 0  \}
	\label{hyperplane_original}
\end{equation}
We can denote $\mathbf{x} = \mathbf{y} + \mathbf{x_{0}}$, hence, equation \ref{hyperplane_original} become:
\begin{equation}
	\mathcal{H} = \{ \mathbf{x} |\, \mathbf{w}^{T}(\mathbf{x} - \mathbf{x_{0}}) = 0 \}
	\label{hyperplane}
\end{equation}
It is clearly that when $\mathbf{x} = \mathbf{x_{0}}$, $\mathbf{w}^{T} (\mathbf{x_{0}} - \mathbf{x_{0}}) = 0$ always hold, hence, $\mathbf{x_{0}}$ is a point within hyper-plane. As a result, equation \ref{hyperplane} can be explained intuitively as: Hyper-plane is a set of any point $\mathbf{x}$ which satisfied the vector $\mathbf{x} - \mathbf{x_{0}}$(start from a fixed point $\mathbf{x_{0}}$ which belong to this hyper-plane, end at $\mathbf{x}$) is perpendicular to a normal vector $\mathbf{w}$.
\end{document}