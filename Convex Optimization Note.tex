\documentclass[10pt,a4paper]{article}
\usepackage[utf8]{inputenc}
\usepackage{amsmath}
\usepackage{amsfonts}
\usepackage{amssymb}
\usepackage[left=2cm,right=2cm,top=2cm,bottom=2cm]{geometry}
\author{Garrick Lin}
\title{Convex Optimization Note}
\setlength{\parindent}{0cm}
\setlength{\parskip}{15pt}
\begin{document}
\maketitle

\section{Affine Set}

\subsection{Definition}
A set $\mathcal{A}$ is called affine set when it satisfied:
\begin{quotation}
	If $\mathbf{x_{1}} \in \mathcal{A}$ and $\mathbf{x_{2}} \in \mathcal{A}$, then, $\forall \theta \in \mathcal{R}$, $x = \theta \mathbf{x_{1}} + (1 - \theta) \mathbf{x_{2}}$ also belong to $\mathcal{A}$. 
\end{quotation}
The expression $\theta \mathbf{x_{1}} + (1 - \theta) \mathbf{x_{2}}$ represent a line that cross through $\mathbf{x_{1}}$ and $\mathbf{x_{2}}$. Hence, affine set can be explained intuitively as:
\begin{quote}
	Affine set is a set that contain the line which cross through any two point within this set.
\end{quote} 

\subsection{Properties}
Assume a affine set $\mathcal{A} \subseteq \mathcal{R}^{n}$. If $\mathcal{A}$ contain original point($\mathbf{0}$), then, $\mathcal{A}$ is a subspace. In order to prove this, $\mathcal{A}$ must satisfied following three rules:  
\begin{enumerate}
	\item $\mathcal{A}$ contain original point.
	\item If $\mathbf{v} \in \mathcal{A}$, then, $\forall \theta \in \mathcal{R}$, $\theta \mathbf{v} \in \mathcal{A}$.
	\item If $\mathbf{v_{1}}, \mathbf{v_{2}} \in \mathcal{A}$, then, $\mathbf{v_{1}} + \mathbf{v_{2}} \in \mathbf{A}$.
\end{enumerate}
The first rule is already satisfied. For the second rule, $\forall \theta \in \mathcal{R}$, $\theta \mathbf{v}$ satisfied:
\begin{equation*}
	\theta \mathbf{v} = \theta \mathbf{v} + (1 - \theta) \mathbf{0}
\end{equation*}
Since $\mathcal{A}$ is affine and $\mathbf{v}, \mathbf{0} \in \mathcal{A}$, $\theta \mathbf{v} \in \mathcal{A}$ always true. For the third rule, there is:
\begin{equation*}
	\mathbf{v_{1}} + \mathbf{v_{2}} = 2(\frac{1}{2} \mathbf{v_{1}} + \frac{1}{2} \mathbf{v_{2}})
\end{equation*}
$\frac{1}{2} \mathbf{v_{1}} + \frac{1}{2} \mathbf{v_{2}}$ is in $\mathcal{A}$ as $\mathcal{A}$ is affine and $\frac{1}{2} \mathbf{v_{1}}, \frac{1}{2} \mathbf{v_{2}} \in \mathcal{A}$(According to the second rule, $\mathbf{v_{1}}, \mathbf{v_{2}} \in \mathcal{A}$, then, $\frac{1}{2} \mathbf{v_{1}}, \frac{1}{2} \mathbf{v_{2}} \in \mathcal{A}$). According  to the second rule, $2(\frac{1}{2} \mathbf{v_{1}} + \frac{1}{2} \mathbf{v_{2}}) \in \mathcal{A}$

Therefore, any affine set $\mathcal{A}'$ can be thought as a subspace $\mathcal{V}$ with a transportation $\mathbf{v}$, which means, $\mathcal{A}' = \mathcal{V} + \mathbf{v}$. On the other hand, an affine set subtract some constant vector ($\mathcal{A}' - \mathbf{v}$), which make the new set contain original point, can form a subspace. Formally, for a affine set $\mathcal{A} = \{ \mathbf{a_{1}}, \mathbf{a_{2}}, \cdots, \mathbf{a_{n}} \}$ and  $\mathbf{a_{i}} \in \mathcal{A}$, the set:
\begin{equation*}
	\mathcal{A} - \mathbf{a_{i}} = \{ \mathbf{a_{1}} - \mathbf{a_{i}}, \mathbf{a_{2}} - \mathbf{a_{i}}, \cdots,  \mathbf{a_{i - 1}}, \mathbf{0}, \mathbf{a_{i + 1}}, \cdots, \mathbf{a_{n}} - \mathbf{a_{i}} \}
\end{equation*}
form a subspace.

\end{document}