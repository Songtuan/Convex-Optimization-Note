\documentclass[10pt,a4paper]{article}
\usepackage[utf8]{inputenc}
\usepackage{amsmath}
\usepackage{amsfonts}
\usepackage{amssymb}
\usepackage[left=2cm,right=2cm,top=2cm,bottom=2cm]{geometry}
\author{Garrick Lin}
\title{Convex Optimization Note}
\setlength{\parindent}{0cm}
\begin{document}
\maketitle

\section{Affine Set}

\subsection{Definition}
A set $\mathcal{A}$ is called affine set when it satisfied:
\begin{quotation}
	If $x_{1} \in \mathcal{A}$ and $x_{2} \in \mathcal{A}$, then, $\forall \theta \in \mathcal{R}$, $x = \theta x_{1} + (1 - \theta)x_{2}$ also belong to $\mathcal{A}$. 
\end{quotation}
The expression $\theta x_{1} + (1 - \theta) x_{2}$ represent a line that cross through $x_{1}$ and $x_{2}$. Hence, affine set can be explained intuitively as:
\begin{quote}
	Affine set is a set that contain the line which cross through any two point within this set.
\end{quote} 

\subsection{Properties}


\end{document}